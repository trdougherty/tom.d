%% 
%% Copyright 2007-2020 Elsevier Ltd
%% 
%% This file is part of the 'Elsarticle Bundle'.
%% ---------------------------------------------
%% 
%% It may be distributed under the conditions of the LaTeX Project Public
%% License, either version 1.2 of this license or (at your option) any
%% later version.  The latest version of this license is in
%%    http://www.latex-project.org/lppl.txt
%% and version 1.2 or later is part of all distributions of LaTeX
%% version 1999/12/01 or later.
%% 
%% The list of all files belonging to the 'Elsarticle Bundle' is
%% given in the file `manifest.txt'.
%% 
%% Template article for Elsevier's document class `elsarticle'
%% with harvard style bibliographic references

%\documentclass[preprint,12pt]{elsarticle}

%% Use the option review to obtain double line spacing
\documentclass[preprint,review,12pt]{elsarticle}

%% Use the options 1p,twocolumn; 3p; 3p,twocolumn; 5p; or 5p,twocolumn
%% for a journal layout:
%% \documentclass[final,1p,times]{elsarticle}
%  \documentclass[final,1p,times,twocolumn]{elsarticle}
%% \documentclass[final,3p,times]{elsarticle}
%% \documentclass[final,3p,times,twocolumn]{elsarticle}
%% \documentclass[final,5p,times]{elsarticle}
%% \documentclass[final,5p,times,twocolumn]{elsarticle}

%% For including figures, graphicx.sty has been loaded in
%% elsarticle.cls. If you prefer to use the old commands
%% please give \usepackage{epsfig}

%% The amssymb package provides various useful mathematical symbols
\usepackage{amssymb}

\usepackage{biblatex} %Imports biblatex package
\addbibresource{applied-energy/library.bib} %Import the bibliography file
%% The amsthm package provides extended theorem environments
%% \usepackage{amsthm}

%% The lineno packages adds line numbers. Start line numbering with
%% \begin{linenumbers}, end it with \end{linenumbers}. Or switch it on
% for the whole article with \linenumbers.
\usepackage{lineno}

\journal{Applied Energy}

\begin{document}

\begin{frontmatter}

%% Title, authors and addresses

%% use the tnoteref command within \title for footnotes;
%% use the tnotetext command for theassociated footnote;
%% use the fnref command within \author or \address for footnotes;
%% use the fntext command for theassociated footnote;
%% use the corref command within \author for corresponding author footnotes;
%% use the cortext command for theassociated footnote;
%% use the ead command for the email address,
%% and the form \ead[url] for the home page:
%% \title{Title\tnoteref{label1}}
%% \tnotetext[label1]{}
%% \author{Name\corref{cor1}\fnref{label2}}
%% \ead{email address}
%% \ead[url]{home page}
%% \fntext[label2]{}
%% \cortext[cor1]{}
%% \affiliation{organization={},
%%             addressline={},
%%             city={},
%%             postcode={},
%%             state={},
%%             country={}}
%% \fntext[label3]{}

\title{}

%% use optional labels to link authors explicitly to addresses:
\author{Thomas Dougherty}
\author{Rishee Jain}
\affiliation{organization={Stanford University},
            addressline={450 Serra Mall},
            city={Stanford},
            postcode={94305},
            state={CA},
            country={USA}}

% \author{Thomas Dougherty}

% \affiliation{organization={Stanford University},%Department and Organization
%             addressline={}, 
%             city={},
%             postcode={}, 
%             state={},
%             country={}}

\begin{abstract}
Urban Building Energy Modeling (UBEM) provides a framework for decabonization decision making on an urban scale. However without consideration of microclimate effects on building energy consumption, the quality of decision making may be impaired. This study provides insight into the potential deterioration of data quality which may arise when the environmental data used is not in close proximity to the buildings of analysis. We then propose an amelioration to this detriment by melting remote sensing data into the analysis, demonstrating the improved building energy prediction quality even at large distances from weather stations.
\end{abstract}

%%Graphical abstract
\begin{graphicalabstract}
%\includegraphics{grabs}
\end{graphicalabstract}

%%Research highlights
\begin{highlights}
\item Research highlight 1
\item Research highlight 2
\end{highlights}

\begin{keyword}
%% keywords here, in the form: keyword \sep keyword

%% PACS codes here, in the form: \PACS code \sep code

%% MSC codes here, in the form: \MSC code \sep code
%% or \MSC[2008] code \sep code (2000 is the default)

\end{keyword}

\end{frontmatter}

%% \linenumbers

%% main text
\begin{linenumbers}
\section{Introduction}
\label{intro}
As buildings consume roughly 40\% of the electricity supply in the United States, they play a key role in decarbonization policy. Urban Building Energy Modeling (UBEM) provides a framework for decabonization decision making, but it has yet to provide high levels of social benefit due to its limited capacity to both accurately model individual building energy use and scale to large urban regions \cite{reinhart_urban_2016}. Current approaches to improve the quality of urban building energy modeling typically focus on the correct assignment of building properties to existing structures \cite{cerezo_comparison_2017}\cite{nagpal_methodology_2019}. This has historically been accomplished through the assignment of building archetypes, which are collected and stored in central datasets like that of CBECS. The assignment of reasonable building archetypes gives the modeler substantially more data for accurate description of building properties, including estimates of fenestration percentage, the type of HVAC equiptment used, or material properties of the walls \cite{dogan_shoeboxer_2017}. New approaches in computer vision have attempted to automate the process of curating building features which may later be used in energy modeling, which may provide a promising avenue for rapid data collection through the use of street level imagery. The improved curation of building properties and the subsequent incorporation of these properties into building energy models will provide new levels of sophistication for policy makers. For example, a higher quality picture of the average window to wall ratio in a city may provide improved quality estimates of energy reduction from upgraded window retrofits.

To bring the benefits of building energy modeling to individual households, the accuracy of energy models at the individual building level must be maintained as the analysis scales to the district or urban level. While individual building properties play a significant role in improving the process of curating high quality estimates for building energy consumption, the consumption of energy in buildings has a delicate relationship with its surroundings. Its adjacent properties such as vegetation, neighboring structures, or impervious surfaces, have been shown to substantially impact the operating performance of the structure \cite{fung_impact_2006}\cite{toparlar_impact_2018}. Macroscopic analysis of Nanjing, China demonstrated that Urban Heat Island (UHI) effects were estimated to increase cooling demand by up to 24\% and increase total energy demand by as much as 5\% \cite{yang_impact_2020}. Another macroscopic study of Tokyo, Japan estimated the temperature sensitivity of peak electricity demand to be 3\%/°C \cite{kikegawa_development_2003}. While wind is less represented in the research for its impact on building energy consumption, high wind speeds modify the heat transfer characteristics of the building with its environment. Preliminary studies have indicated that wind may increase the energy consumption of a structure by as much as 5\% \cite{khoshdel_nikkho_quantifying_2017}. These works signal the potential benefit of high resolution urban climate models, which have been a fixture of the research community for the past two decades, starting with urban canopy models \cite{}. These complex interactions are often distilled into a resistive-capacitive (RC) system, which enables the modeler to represent interactions using an electrical equivalence \cite{bueno_resistance-capacitance_2012} \cite{nouvel_simstadt_2015}. Modern research into urban climates now provides utilities to generate files which may be used for more accurate urban building energy modeling \cite{bueno_urban_2013}. 

Recent work has highlighted existing disconnects between climate modeling and energy systems \cite{javanroodi_impacts_2019}\cite{craig_overcoming_2022}. This crituque is particularly cutting when considering the existing tools for UBEM, which typically rely on Typical Meteorological Year (TMY) files for the climate conditions of the structure. These files are generated from nearby weather stations, which may or may not be in the same microclimate zones as the simulated buildings. However, given that climate change is likely to impact urban areas more significantly than that of nearby rural regions \cite{huang_projecting_2019}, it is unlikely that TMY files will be a more accurate representation of the climate moving forward.

While work has been conducted to generate utilities for modeling urban microclimate conditions \cite{kusaka_simple_2001}\cite{bueno_urban_2013}\cite{romero_rodriguez_urban-scale_2020}, typically it is done so with the goal of exploring health consequences of UHI \cite{hsu_disproportionate_2021}, pedestrian comfort \cite{allegrini_influence_2015}, or water resources \cite{de_ridder_urbclim_2015}. This work seeks to shed light on the potential benefits of including hyperlocalized climate variables in building energy modeling.

\section{Data}
This anlysis can be thought of as a regression problem, with the task of predicting the energy consumption of the structure. The energy of large buildings is provided by New York City on a monthly interval between Janurary 2018 and December 2020. The building attributes used in this analysis are provided by the building footprints dataset of New York, which provides a geometric representation of the buildings' outlines in geojson format in addition to the height and age of the structure. The geometry of the building is provided in World Geodetic System 1984 format, which does not preserve distances when used in traditional measures. Therefore the area of the building was computed by first projecting the buildings into UTM Zone 18N, which uses meters for its coordinate systems and preserves the quality of measures between the longitudes of 78/°W and 72/°W.

Two environmental datasets were used in this analysis. The first parallels that of the study conducted by Dougherty et al. \cite{}, which collects precision microclimate features in the immediate region around the structure from a variety of remote sensing sourcs. These sources include Sentinel-2, NOAA reanalysis, elevation maps, and night time imagery of the city. To match the high temporal resolution of this remote sensing data with that of the monthly energy data, the variables captured by remote sensing are first averaged on the monthly level. 

The second data set used in the analysis was curated by extracting envirionmental features from EnergyPlus Weather (EPW) files, which are used synonymously with TMY files. In an attempt to capture more pertinent weather station for each building, we first constructed a system to identify the nearest weather stations to each structure. A data set of available weather stations in the United States is provided by NREL (CITE THIS https://data.nrel.gov/submissions/156), which contains not only the location of each weather station in the United States but an associated EPW file curated from historical weather data at the station. The data types are mapped according to the specification laid out in Chapter 2 of the EnergyPlus "Auxillary Programs" documentation, titled "Weather Converter Program".

In a pipeline to reduce the overall computation required for distance measurements, a set of candidate weather stations were curated by first computing the centroid of all buildings in the analysis and projecting a uniform radius of 45km around this centroid. This radius is chosen based on the area of New York City, which is recorded to be roughly 783.8 sq.km, and on visible inspection to confirm that all points of the data are adequately captured by the radius.

\begin{figure}[h]
    \includegraphics[width=8cm]{applied-energy/images/weather_stations_map.png}
\end{figure}

The distance between each point and each weather station is then computed, with the smallest distance serving as an indicator of the mapping between the building and the weather station. Given that prior research has employed a similar distance metric to curate a high resolution model of urban climate \cite{hong_urban_2021}, we feel this to be an appropriate method of mapping buildings to weather stations.

\begin{figure}[h]
    \includegraphics[width=8cm]{applied-energy/images/building_station_membership.png}
\end{figure}

Additionally, this sheds insight into the typical relationship between buildings and their distances to the nearest weather stations


~~ maybe put in a table here of the variables used or captured by the EPW files.

\section{Methods}



\begin{figure}[h]
    \includegraphics[width=8cm]{applied-energy/images/building_distances.png}
\end{figure}

\section{Methods}

\end{linenumbers}


%% The Appendices part is started with the command \appendix;
%% appendix sections are then done as normal sections
%% \appendix

%% \section{}
%% \label{}

%% For citations use: 
%%       \citet{<label>} ==> Jones et al. [21]
%%       \citep{<label>} ==> [21]
%%

%% If you have bibdatabase file and want bibtex to generate the
%% bibitems, please use
%%
\printbibliography %Prints bibliography

%% else use the following coding to input the bibitems directly in the
%% TeX file.

%% \begin{thebibliography}{00}

%% \bibitem[Author(year)]{label}
%% Text of bibliographic item
\end{document}

\endinput
%%
%% End of file `elsarticle-template-num-names.tex'.
